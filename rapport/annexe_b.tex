\noindent
Le fichier \texttt{ast\_repr.ml} contient la définition de l'arbre de
syntaxe abstraite d'un composant Scade après avoir été parsé.
\begin{small}
\begin{verbatim}

type ident = string

type value =
  Bool of bool
| Int of int
| Float of float

type base_type =
  T_Bool
| T_Int
| T_Float

(* Opérateurs binaires de Scade *)
type bop =
  Op_eq | Op_neq | Op_lt | Op_le | Op_gt | Op_ge
| Op_add | Op_sub | Op_mul | Op_div | Op_mod
| Op_div_f | Op_and | Op_or | Op_xor

(* Opérateurs unaires de Scade *)
type unop =
  Op_not | Op_minus

(* Expressions Scade *)
type p_expression =
  PE_Ident of ident
| PE_Value of value
| PE_Array of p_array_expr
| PE_App of ident * p_expression list
| PE_Bop of bop * p_expression * p_expression
| PE_Unop of unop * p_expression
| PE_Fby of p_expression * p_expression * p_expression
| PE_If of p_expression * p_expression * p_expression

(* Opérations sur les tableaux*)
and p_array_expr =
  PA_Def of p_expression list (* Définition avec la syntaxe [e1, ..., en] *)
| PA_Caret of p_expression * p_expression   (* Définition avec la syntaxe e1^e2 (ex: false^4) *)
| PA_Concat of p_expression * p_expression
| PA_Slice of ident * (p_expression * p_expression) list
| PA_Index of ident * p_expression list
| PA_Reverse of p_expression

(* Partie gauche d'une équation *)
type p_left_part =
  PLP_Ident of ident
| PLP_Tuple of ident list

(* Equation, une partie gauche et une expression à droite *)
type p_equation = p_left_part * p_expression

(* Type d'une variable (base ou tableau) *)
type p_type =
  PT_Base of base_type
| PT_Array of p_type * p_expression

(* Déclaration d'une variable *)
type p_decl = ident * p_type

(* Node, défini par un identifiant, des paramètres d'entrées/sorties, les
conditions associées, une liste de variables locales, et une liste d'équations *)
type p_node =
  {  p_id: ident;
     p_param_in: p_decl list;
     p_param_out: p_decl list;
     p_assumes: p_expression list;
     p_guarantees: p_expression list;
     p_vars: p_decl list;
     p_eqs: p_equation list; 
  }

\end{verbatim}
\end{small}
