% -*- coding: utf-8; ispell-dictionary: "french"

% Introduction

Ce stage s'est déroulé au sein d'un projet financé par l'Agence
Nationale de la Recherche: CERCLES2. Ce projet a pour but la
certification compositionnelle des logiciels embarqués critiques et
sûrs. C'est la notion de composant réutilisable et assemblable pour former des
logiciels critiques et sûrs qui est à la base du projet, l'intérêt étant à la fois
pratique par le gain de temps et d'effort, et économique. \\

Pour assurer que ces composants sont sûrs et réutilisables, on
utilise une méthode formelle, qui permet d'exprimer la signification d'un
composant dans un formalisme mathématique. Il faut alors introduire le concept des
contrats: un contrat est un composant associé à des conditions sur ses entrées
(pré-conditions) et sur ses sorties (post-conditions). 
A partir d'un composant et de son contrat, il faut alors vérifier formellement
que:
\begin{itemize}
\item  (i) la définition du composant satisfait le contrat
\item (ii) l'utilisation du composant satisfait les pré-condition, et en
conséquence de (i) le résultat satisfait les post-conditions.
\end{itemize}
La validation est alors faite par une démonstration formelle.

Un acteur majeur du développement de systèmes embarquées critiques est Scade,
un acronyme pour Safety Critical Application Developpement Environment. Cet
environnement de développement est basé sur la programmation graphique, par
schémas-blocs, permettant de définir des programmes faciles à lires et
permettant d'engendrer directement du code compilable (C ou ADA). Il est
notamment utilisé en aéronautique (grande partie du logiciel embarqué de
l'A380), dans le domaine spatial ou dans le nucléaire. C'est donc avec Scade
que sont écrits les composants, les contrats étants rédigés sous forme
textuelle en accompagnement du composant.\\

INSERER CAPTURE SCADE\\

Concernant la méthode formelle, c'est la méthode B qui est utilisée. Introduite
par J.R. Abrial dans les années 80, elle est basé sur le raffinement de
spécifications formelles vers une spécification exécutable. La spécification
formelle est rédigée dans un formalisme mathématique de haut niveau appelé
machine abstraite. Le raffinement de cette machine abstraite consiste à la
reformuler de façon plus concrète et à l'enrichir avec des substitutions
correspondants aux instructions du composant. Ce raffinement de plus bas niveau
est appelé implantation. Chaque étape de raffinement passe par une étape
d'obligation de preuve, une validation par démonstration formelle, garantissant
la fidélité de la spécification raffinée par rapport à la spécification
originale. \\
L'avantage de la méthode B est qu'elle est permet la composition de programme,
et ainsi permet de faire référence à des composants déjà certifiés. De plus
cette méthode a déjà fait ses preuves industriellement, elle a notamment été
utilisée pour développer la ligne METEOR (ligne 14) du métro parisien, qui est
entièrement automatisée.\\

Mon travail fut de développer un traducteur permettant de passer d'une méthode à
l'autre. Le traducteur suit une ligne de compilation classique, prenant en
entrée un code issu de Scade et produisant en sortie une machine abstraite
correspondant aux spécification du contrat, ainsi que la machine raffinée qui
implante le composants. \\

SCHEMA PRINCIPE GENERAL
